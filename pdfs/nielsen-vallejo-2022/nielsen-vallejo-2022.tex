\documentclass[twoside]{article}
\usepackage{mathpazo}
\usepackage[T1]{fontenc}
\usepackage[utf8]{inputenc}
\usepackage{graphicx}
\usepackage[colorlinks]{hyperref}
\hypersetup{
  colorlinks,
  urlcolor=blue,
  linkcolor=black
}
\usepackage{caption} % unnumbered video examples
% \usepackage{authblk} % author affiliations
\usepackage{hanging} % hanging indents

% headers
\usepackage{fancyhdr}
\renewcommand{\headrulewidth}{0pt}

% first page footer
\fancypagestyle{infofooter}{%
  \fancyhf{}
  \renewcommand\headrulewidth{0pt}
  \fancyfoot[L]{\sffamily\small 
    WORLD MUSIC TEXTBOOK, ISSN: 2767-4215; \copyright~2022, CC-BY-NC-ND\\
    https://doi.org/10.25035/XXXXX}
}
\thispagestyle{infofooter} % remove header from first page
\pagestyle{fancy}     % add headers in other pages

% normal headers
\fancyhead[LO]{\sffamily\small \textbf{\thepage} \quad World Music Textbook}
\fancyhead[RE]{\sffamily\small Nielsen and Vallejo: Music and Nationalism \quad \textbf{\thepage}}
\fancyfoot{}

% flush left title
\makeatletter
\renewcommand{\maketitle}{\bgroup\setlength{\parindent}{0pt}
\begin{flushleft}
  \vspace*{3\baselineskip}
  \huge{\textbf{\@title}}

  \medskip
  
  \large{\@author}
\end{flushleft}\egroup
}
\makeatother

% for keywords and abstract
\providecommand{\abstracttext}[1]
{
  \noindent
  \textbf{Abstract:} #1
}

\providecommand{\keywords}[1]
{
  \newline
  \textbf{Keywords:} #1
}

% for link
\providecommand{\wmturl}{\href{https://worldmusictextbook.org/nielsen-vallejo-2022}{https://worldmusictextbook.org/nielsen-vallejo-2022}}
\providecommand{\wmturltext}{
  \noindent\emph{The online version of this chapter includes all embedded content and is available at \wmturl.}
}

% metadata
\title{Music and Nationalism}
\author{Kristina F. Nielsen | Southern Methodist University\\Jessie Vallejo | Cal Poly Pomona}
% \affil{}

\date{}

% document
\begin{document}
\suppressfloats % prevent float above title
\maketitle

\abstracttext{This article offers an overview of musical nationalism. It considers how states have used music as a political tool as well as the ways in which communities have employed music to reject national identities and challenge nation-states.}
\keywords{activism, global, immigration, Music Notes, nationalism, politics}

\smallskip

\wmturltext

\medskip

\noindent\hfil\rule{0.5\textwidth}{0.4pt}\hfil

\bigskip

\noindent\textbf{Nationalism:} \emph{The concept that people, lands, and cultures
are divided into nations that provide a central component of citizens'
identities.}

\bigskip

Modern nationalism developed in Western Europe in the revolutionary
atmosphere of the late eighteenth century. During this time, the
nation-state became increasingly central to defining territories and the
peoples who inhabited them.\footnote{A nation-state is defined as an
  entity in which a majority of citizens identify with each other
  culturally. It retains autonomy over the corresponding land inhabited
  by those who identify as its citizens.} Nationalism seeks to
homogenize and unite those within a territory, often antagonizing
individuals within its borders who do not conform and are therefore
considered a threat to a perceived national unity.

\sloppy
Nationalist movements in the 1800s sought to create musics to reflect
the desired ideals of nations. This imperative emerged as European
nations increasingly sought to define themselves in opposition to each
other and the growing number of colonies. European nationalist composers
often drew on melodic or rhythmic traits considered particularly
indicative of national styles. Because music is a symbolic medium, the
meaning of these rhythms and melodies was culturally established:
Listeners and composers were themselves cultivated to hear certain
traits as nationalist, developing these musical symbols of nationalism
(Beckerman 1986:73). Many nationalist composers have drawn on collected
folk culture and music from across territories and peoples subsumed into
nations. These folk collections are presented as a shared national
heritage or folk culture. In many cases, these projects claimed to
``save'' or ``preserve'' culture, often by notating versions of oral
traditions in written forms, effectively freezing them into one single
``official'' version. Music books and media, such as radio, recordings
and film, aided the dissemination of a shared national culture that
sought to cultivate national identity as a dominant force in citizens'
lives (See Anderson 1983; Boyes 2010).

In formerly-colonized lands, nationalism reflects a complicated mixture
of both liberation and the continuation of colonial power structures.
With the advent of independence movements, lands that were previously
colonized by European nations have sought to create their own national
identities, including musical identities. In doing so, they have
frequently drawn on the same model as European nations that push a
homogenized cultural form as a shared national heritage. Following the
European model of turning to ``the folk,'' or communities seen as
preserving older strata of culture, many countries have appropriated
these musics as sources of national music heritage. In countries like
Mexico, Peru, and Ecuador, the music of Indigenous communities---many of
whom remain wary of state governments because of histories of
genocide---has been co-opted into indigenismo nation-building projects
in the twentieth century.\footnote{A political strategy to co-opt
  Indigenous culture int nation-states while excluding Indigenous
  people.} One famous example is ``El Cóndor Pasa,'' a piece from the
eponymous zarzuela (operetta) composed in 1913 by Peruvian composer and
ethnomusicologist Daniel Alomía Robles. Similar efforts have taken place
in African countries where governmental organizations have played
significant roles in shaping a shared national culture in the second
half of the twentieth century when many African nations gained
independence. In countries like Ghana, government organizations have
supported performances with different ethnic groups representing
cultures across its territories (Agawu 2003:19). These efforts face the
challenges of representation: What music represents diverse nations of
people, often speaking many different languages? Who gets to decide?

Settler colonial states---meaning the dominant population is descended from colonizers who have oppressed Indigenous communities through acts of genocide, systematic dispossession of lands, and cultural repression---have also used music to define themselves. These nation-states, including the United States, Canada, Australia, and South Africa, may recognize some differences among groups of people when those differences do not challenge the nationalist rhetoric. Yet overall, they demand assimilation and depend on native societies losing their land, language, and cultural practices (Simpson 2014; Wolfe 2006).

Political and social figures have showcased---and in many cases, misrepresented---musical styles and dances that they feel should be celebrated as ``roots'' music representing their idea of a nation. For example, Appalachian old-time and country music in the United States has been overwhelmingly portrayed as Anglo- and Euro-American music; the African and Black roots of these musical styles were intentionally down-played and erased in part through the efforts of wealthy Anglo-Americans such as Henry Ford (See Brucher 2016).

So how do communities and musicians respond to nations and ideas of
national musics when these same musicians often find themselves
systematically excluded? In some cases, people push for inclusion and
recognition, finding spaces for marginalized identities to be viewed and
heard as part of a national heritage. This approach might be observed
when the lyrics of a national anthem includes several languages; for
instance, South Africa changed their national anthem following apartheid
to better reflect the inclusive ideals of a new era (see example below).
Research and scholarship have also played an important role in amending
histories that have marginalized communities (See Hay 2003; Flemons
2018; Our Native Daughters 2019). Another example is mariachi music,
which has been embraced as a representative of Mexican and Latin
American cultures in the United States and public-school music
curricula. At the same time, it can also be understood as a vital U.S.
American music given its significance within the United States today
(See Salazar 2011; Sheehy 2006).

In other cases, various racial, ethnic, and economic social groups
within a nation may disagree, redefine, or create their own definitions
of what their national music is (Wong 2012). And whereas it may be
assumed that membership within a nation is or should be an end goal for
those who are marginalized and seeking the benefits of citizenship,
there are contexts in which rejecting membership of one nation may be
preferred in order to advocate for the sovereignty of another identity
group. Music, along with other culturally-identifying practices such as
language and dance, have often been central for people resisting or
rejecting a nationalist identity during the twentieth and twenty-first
centuries, such as the second Hawai'ian Renaissance Movement's protest
songs and renewed interests in Hawai'ian language, chant and dance
(Lewis 1987; Stillman 1995:4-5).

In summary, although nationalism in its current form is a relatively
recent phenomenon that began among colonial European nations, today it
structures assumptions of how humans organize themselves, cultures, and
musics. While music may create affinity within nations, musical
nationalism does so by excluding others---often by erasing communities
and practices all together.

\hypertarget{additional-materials}{%
\section*{Additional Materials}\label{additional-materials}}

\emph{Note: this chapter has an \href{https://worldmusictextbook.org/nielsen-vallejo-2022}{accompanying annotated
playlist available here}.}

\hypertarget{discussion-questions}{%
\subsection*{Discussion Questions}\label{discussion-questions}}

\begin{enumerate}
\def\labelenumi{\arabic{enumi}.}
\item
  In the passage above, it is mentioned that folklore projects ``claimed
  to `save' or `preserve' culture, often by documenting or freezing
  versions of oral traditions in written forms.'' In many cases,
  however, there are multiple versions of songs, meaning that decisions
  must be made about which versions to preserve or make ``official'' by
  publishing. What problems might arise in this process? Who should get
  to make these decisions about which songs join folk music canons and
  which ones are left out? What might be some consequences of these
  choices?
\item
  Are people able to claim an identity or membership within a nation or
  political grouping even if they are not recognized as a member from
  the nation or political grouping? Why or why not? What are some of the
  identity issues that arise when one's claims to membership may or may
  not be recognized by a larger group? How might these discrepancies
  impact musical performance?
\item
  Make a list of rights or benefits that are awarded to someone who is
  able to claim citizenship to a nation state. Then consider what is at
  stake for someone who is denied citizenship to a nation state.
\item
  Conduct an internet search using Google or a database like JSTOR to
  read about plurinationalism. Find an example of how this is practiced
  in a given nation state or country. What are some of the social
  challenge of this political structure? How does music relate to the
  ways this plurinational state is represented?
\end{enumerate}

\hypertarget{recommended-readings}{%
\subsection*{Recommended Readings}\label{recommended-readings}}

\begin{hangparas}{15pt}{1}
  \emph{Black Music Research Journal.} 2003. \href{https://www.jstor.org/stable/i369762}{Special issue on  Affrilachian music} 23(1). .

  Bohlman, Philip V. 2010.~\emph{Focus: Music, Nationalism, and
  the Making of the New Europe: Music, Nationalism, and the Making of the
  New Europe}, Second Edition. New York: Routledge.

  Einarsdóttir, Áslaug, and Valdimar Hafstein. 2018. \href{http://flightofthecondorfilm.com/}{\emph{The
  Flight of the Condor: A Letter, a Song and the Story of Intangible
  Cultural Heritage}}. Indiana University Press. Film.

  Guilbault, Jocelyne. 2007. \emph{Governing Sound: The
  Cultural Politics of Trinidad's Carnival Musics}. Chicago: The
  University of Chicago Press.

  Tang, Kai. 2021. "Singing a Chinese Nation: Heritage Preservation, the Yushengtai Movement, and New Trends in Chinese Folk Music in the Twenty-First Century. \emph{Ethnomusicology} 65(1):1-31.

  Taruskin, Richard. 2001. ``Nationalism.'' \emph{Grove Music
  Online}. Oxford University Press.
\end{hangparas}

\bigskip
\hypertarget{recommended-media}{%
\subsection*{Recommended Media}\label{recommended-media}}

\begin{hangparas}{15pt}{1}
  Alomía Robles, Daniel. 1913. \href{https://www.youtube.com/watch?v=Rvp-RFpgXA0}{Orchestral recording based on the original scores}.

  Flemons, Dom. 2018. \href{https://folkways.si.edu/dom-flemons/black-cowboys}{\emph{Dom Flemons Presents: Black Cowboys.}} Washington, DC: Smithsonian Folkways Recordings SFW40224.

  Kiona, Kaulaheaonamiku. 1962. \href{https://folkways.si.edu/kaulaheaonamiku-kiona/hawaiian-chant-hula-and-music/hawaii/music/album/smithsonian}{\emph{Hawaiian Chant, Hula and Music.}} Washington, DC: Folkways FW08750.
    
  \href{https://youtu.be/ezTKNGGJuto}{Mariachi Los Gavilanes de Monaco Middle School -- Heart of Las Vegas Television Documentary.}

  Pop-Up Magazine. 2020. \href{https://www.youtube.com/watch?v=RXywZ2CXcnk}{"Going Varsity in Mariachi."}
  Documentary video.

  Our Native Daughters. 2019. \href{https://folkways.si.edu/songs-of-our-native-daughters}{\emph{Songs of Our Native Daughters.}} Washington, DC: Smithsonian Folkways Recordings SFW40232.
  
  Various Artists. 1989. \href{https://folkways.si.edu/musics-of-hawaii-anthology-of-hawaiian-music-special-festival-edition/album/smithsonian}{\emph{Musics of Hawai'i: Anthology of Hawaiian Music - Special Festival Edition}}. Washington DC: Smithsonian Folkways Recordings SFW40016.
\end{hangparas}

\bigskip
\hypertarget{works-cited}{%
\section*{Works Cited}\label{works-cited}}

\begin{hangparas}{15pt}{1}
  Anderson, Benedict. 1983 {[}1991{]}. \emph{Imagined
  Communities: Reflections on the Origin and Spread of Nationalism}.
  London: Verso.

  Agawu, Kofi. 2003. \emph{Representing African Music:
  Postcolonial Notes, Queries, Positions}. New York: Routledge.

  Beckerman, Michael. 1986. ``In Search of Czechness.''
  \emph{19th-Century Music} 10(1):61-73.

  Boyes, Georgina. 2010 {[}2012{]}. \emph{The Imagined Village:
  Culture, Ideology and the English Folk Revival.} Leeds: Manchester
  University Press.

  Brucher, Katherine. 2016. "Assembly Lines and Contra Dance
  Lines: The Ford Motor Company Music Department and Leisure Reform."
  \emph{Journal of the Society for American Music} 10(4):470-495.

  Hay, Fred J. 2003. "Black Musicians in Appalachia: An
  Introduction to Affrilachian Music." \emph{Black Music Research Journal}
  23(1/2):1-19.

  Lewis, George H. 1987. "Style in Revolt Music, Social
  Protest, and the Hawaiian Cultural Renaissance." \emph{International
  Social Science Review} 62(4):168-177.

  Salazar, Lauryn C. 2011. "From Fiesta to Festival: Mariachi
  Music in California and the Southwestern United States."
  Ph.D.~dissertation, University of California, Los Angeles.

  Sheehy, Daniel E. 2006. \emph{Mariachi Music in America:
  Experiencing Music, Expressing Culture}. New York: Oxford University
  Press.

  Simpson, Audra. 2014. \emph{Mohawk Interruptus: Political
  Life Across the Borders of Settler States.} Durham, NC: Duke University
  Press.

  Stillman, Amy Ku'uleialoha. 1995. "Not All Hula Songs Are
  Created Equal: Reading the Historical Nature of Repertoire in
  Polynesia." \emph{Yearbook for Traditional Music} 27:1-12.

  Wolfe, Patrick. 2006. "Settler Colonialism and the
  Elimination of the Native." \emph{Journal of Genocide Research}
  8(4):387-409.

  Wong, Ketty. 2012. \emph{Whose National Music?: Identity,
  Mestizaje, and Migration in Ecuador}.
\end{hangparas}
\end{document}